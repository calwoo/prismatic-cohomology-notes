\documentclass[12pt]{amsproc}   
\usepackage[margin=1in, top=1in, bottom=1in]{geometry}                		
\geometry{letterpaper} 
\usepackage{amsthm}                  		
\usepackage{graphicx}	
\usepackage[mathscr]{euscript}	
\usepackage{amssymb, amsmath}
\usepackage{amscd}
\usepackage{tikz-cd}
\usepackage{xcolor}
\usepackage{csquotes}

%theorem styles
\newtheorem*{thm}{Theorem}
\newtheorem*{prop}{Proposition}
\newtheorem{lemma}{Lemma}
\newtheorem*{cor}{Corollary}

\theoremstyle{definition}
\newtheorem*{definition}{Definition}
\newtheorem*{example}{Example}
\newtheorem*{remark}{Remark}
\newtheorem*{notation}{Notation}

%notational macros
\newcommand{\A}{\mathscr{A}}
\newcommand{\sph}{\mathbb{S}}
\newcommand{\G}{\mathbb{G}}
\newcommand{\R}{\mathbb{R}}
\newcommand{\RP}{\mathbb{R}P}
\newcommand{\Q}{\mathbb{Q}}
\newcommand{\Z}{\mathbb{Z}}
\newcommand{\N}{\mathbb{N}}
\newcommand{\E}{\mathbb{E}}
\newcommand{\m}{\mathfrak{m}}
\DeclareMathOperator{\pt}{pt}
\DeclareMathOperator{\BGL}{BGL}
\DeclareMathOperator{\Bcrys}{B_{crys}}
\newcommand{\eff}{\mathcal{F}}
\newcommand{\tensor}{\otimes}
\newcommand{\cotensor}{\Box}
\newcommand{\sma}{\wedge}
\newcommand{\Ztwo}{\mathbb{Z}/(2)}
\newcommand{\iso}{\cong}
\newcommand{\htpyeq}{\simeq}
\newcommand{\coproduct}{\amalg}
\newcommand{\bd}{\textbf}
\newcommand{\T}{\mathbb{T}}
\def\*{\bullet}

%arrows
\newcommand{\ra}{\rightarrow}
\newcommand{\lra}{\longrightarrow}
\newcommand{\xra}{\xrightarrow}
\newcommand{\inja}{\rightarrowtail}
\newcommand{\surja}{\twoheadrightarrow}
\newcommand{\isoarrow}{\xrightarrow{\sim}}

%new operators
\DeclareMathOperator{\Hom}{hom}
\DeclareMathOperator{\Ext}{Ext}
\DeclareMathOperator{\im}{im}
\DeclareMathOperator{\coker}{coker}
\DeclareMathOperator{\h}{H}
\DeclareMathOperator{\id}{id}
\DeclareMathOperator{\Nat}{Nat}
\DeclareMathOperator{\Spec}{Spec}
\DeclareMathOperator{\THH}{THH}
\DeclareMathOperator{\HH}{HH}
\DeclareMathOperator{\TC}{TC}
\DeclareMathOperator{\TR}{TR}
\DeclareMathOperator{\Hdr}{H_{dR}^*	}
\DeclareMathOperator{\Hcrys}{H_{crys}^*}
\DeclareMathOperator{\Het}{H_{et}^*}
\DeclareMathOperator{\Hlogcrys}{H_{log-crys}^*}
\DeclareMathOperator{\dlog}{dlog}
\DeclareMathOperator{\invlim}{\varprojlim}
\DeclareMathOperator{\holim}{holim}
\DeclareMathOperator{\dirlim}{\varinjlim}
\DeclareMathOperator{\colim}{colim}
\DeclareMathOperator{\hocolim}{hocolim}
\DeclareMathOperator{\Fil}{Fil}
\newcommand{\teich}{\underline}
\newcommand{\pcomp}[1]{{#1}^{\sma}_p}
\DeclareMathOperator{\denom}{denom}
\DeclareMathOperator{\JRad}{JRad}

%functor names
\DeclareMathOperator{\Tan}{\mathscr{T}an}
\DeclareMathOperator{\Oh}{\mathscr{O}}
\DeclareMathOperator{\F}{\mathbb{F}}
\DeclareMathOperator{\Fr}{F}
\DeclareMathOperator{\V}{V}
\DeclareMathOperator{\Der}{\mathbf{Der}}
\DeclareMathOperator{\Inf}{Inf}
\DeclareMathOperator{\gr}{gr}
\DeclareMathOperator{\Sym}{Sym}
\DeclareMathOperator{\GL}{GL}
\DeclareMathOperator{\Fun}{Fun}
\DeclareMathOperator{\Iso}{iso}
\DeclareMathOperator{\K}{K}
\DeclareMathOperator{\DerivedL}{\mathbb{L}}
\DeclareMathOperator{\Gal}{Gal}
\DeclareMathOperator{\Map}{Map}
\DeclareMathOperator{\Emb}{Emb}
\DeclareMathOperator{\DPenv}{DP-env}
\DeclareMathOperator{\logDPenv}{log-DP-env}

%category names
\newcommand{\C}{\mathscr{C}}
\newcommand{\D}{\mathscr{D}}
\newcommand{\catgrmod}{\mathscr{G}r\mathscr{M}od}
\newcommand{\catmod}{\textbf{Mod}}
\newcommand{\catcomod}{\mathscr{M}od^{\textbf{co}}}
\newcommand{\catring}{\textbf{Ring}}
\newcommand{\catalg}{\textbf{Alg}}
\newcommand{\catset}{\textbf{Set}}
\newcommand{\cataff}{\textbf{Aff}}
\newcommand{\Top}{\textbf{Top}}
\newcommand{\Mfld}{\text{Mfld}}
\newcommand{\Aff}{\textbf{Aff}}
\newcommand{\Sp}{\text{Sp}}
\newcommand{\Spc}{\text{Spc}}
\newcommand{\Ealg}[1]{\mathbb{E}_{#1}\textbf{-Alg}}
\newcommand{\Einfalg}{\mathbb{E}_{\infty}\textbf{-Alg}}
\newcommand{\catsetfin}{\textbf{Set}_\text{fin}}
\DeclareMathOperator{\Ring}{Ring}


%other macros
\newcommand{\todo}[1]{\textcolor{red}{\textbf{TODO:}\text{[#1]}}}
\newcommand{\citeme}[1]{\textcolor{blue}{\textbf{CITE ME:}\text{[#1]}}}
\newcommand{\remove}[1]{\textcolor{orange}{\textbf{REMOVE?:}\text{[#1]}}}
\newcommand{\ddd}{\cdot\cdot\cdot}


%\title{$\delta$-rings}
%\author{Calvin D. Woo}
\begin{document}
%\maketitle

This document contains my notes on integral perfectoid rings. Let $p$ be fixed throughout.

\section{perfectoid fields}
\begin{definition} A \textbf{perfectoid field} is a field $K$ equipped with a nonarchimedean absolute value $|\cdot|_K : K\ra \R_{\ge 0}$ satisfying
\begin{enumerate}
\item The residue field $k=\Oh_K/\m_K$ has characteristic $p$. 
\item $K$ is complete with respect to the valuation.
\item The Frobenius map $\varphi: \Oh_K/p\Oh_K \ra \Oh_K/p\Oh_K$ is surjective.
\item The maximal ideal $\m_K$ is not generated by $p$, i.e. there is a \textbf{pseudo-uniformizer} $\pi\in K$ so that $|p|_K < |\pi|_K < 1$.
\end{enumerate}
\end{definition}

Note that, if $K$ were itself of characteristic $p$, then the third axiom just states that $K$ is a completely valued perfect field of characteristic $p$. An easy way to check if a completely valued field is perfectoid is to check if each element $x\in K$ has a $p$-th root. This gives us loads of examples.

\begin{example}
Let $\Z_p^{\text{cyc}}$ by the $p$-completion of the union $\bigcup_n \Z[\zeta_{p^n}]$. The maximal $p$-cyclotomic extension of $\Q_p$ is given by $\Q_p^{\text{cyc}}=\Z_p^{\text{cyc}}[1/p]$. Then $K=\Q_p^{\text{cyc}}$ is a perfectoid field with ring of integers $\Oh_K=\Z^\text{cyc}_p$.
\end{example}

Let $K$ be a completely valued field of residue characteristic $p$. We define the \textbf{tilt} of $K$ to be the inverse limit of the system
\[	\cdots \ra K \xra{x\mapsto x^p} K \xra{x\mapsto x^p} K 	\]
We denote the tilt by an annotation $K^\flat$. The elements can be identified by Frobenius-compatible sequences of elements in $K$. There is an obvious multiplication
\[	\{x_n\}_{n\ge 0}\cdot \{y_n\}_{n\ge 0} = \{x_n\cdot y_n\}_{n\ge 0}	\]
which makes $K^\flat$ into a commutative monoid.

The resulting monoid has a "valuation" (not yet) given by $|\{x_n\}_{n\ge 0}|_{K^\flat}=|x_0|_K$. We denote this map by the sharp annotation $\{x_n\}_{n\ge 0}^\sharp=x_0$. In particular, we can define the subset $\Oh_{K^\flat}\subset K^\flat$ to be those sequences where each $x_n\in\Oh_K$
\[	\Oh_{K^\flat}\htpyeq \invlim(\cdots\ra \Oh_K\xra{x\mapsto x^p}\Oh_K)	\]
This statement might only make sense as monoids for now. But really this can be upgraded to an equivalence of rings.

\begin{prop} Let $K$ be a completely valued field of residue characteristic $p$. Then the canonical map induces a bijection
\[	\Oh_{K^\flat}\xra{\htpyeq} \invlim(\cdots\ra \Oh_K/p\Oh_K\xra{x\mapsto x^p}\Oh_K/p\Oh_K) \]
\end{prop}

The proof relies on the elementary observation that for $x,y\in \Oh_K$, we have 
\[ x\equiv y\mod p^{n-1} \implies x^p\equiv y^p\mod p^n\]
As a consequence, we can endow $\Oh_K^\flat$ (and hence $K^\flat$) with the structure of a commutative ring.

So far, we didn't use the fact that $K$ was perfectoid to show that the tilt $K^\flat$ is a field. However, usually the tilt is very small.

\begin{example} For $K=\Q_p$, the tilt is given by $K^\flat=\F_p$.
\end{example}

The sharp map $\sharp:K^\flat \ra K$ gives us a comparison between $K$ and its tilt. Indeed, it's not a ring map because the two have drastically different characteristics! 

As a last remark on perfectoid fields, we note that if $K$ was a perfectoid field, so is the tilt $K^\flat$ a perfectoid field (but this time of characteristic p).

\section{integral perfectoid rings}

It seemed that the properties of perfectoid fields come from their rings of integers. Their properties can be formalized into the concept of an integral perfectoid ring.

\begin{definition} A commutative ring $R$ is \textbf{integral perfectoid} if there is a non-zerodivisor $\pi\in R$ so that
\begin{enumerate}
\item $A$ is p-adically complete.
\item $\pi^p=pu$ for some unit $u\in R$.
\item The Frobenius $R/p\ra R/p$ is surjective.
\item The kernel of Fontaine's map $\theta:A_\text{inf}(R)\ra R$ is principal.
\end{enumerate}
\end{definition}

We call such a $\pi$ a pseudo-uniformizer. We should explain what $A_\text{inf}(R)$ is. As the Frobenius is surjective we get a Frobenius structure on the tilt of $R$
\[	R^\flat=\invlim(\cdots\ra R/p\xra{x\mapsto x^p} R/p)	\]
that turns $R^\flat$ into a perfect ring. Applying the Witt vector construction we get $A_\text{inf}(R)=W(R^\flat)$. The canonical map $R^\flat\ra R/p$ can be lifted by virtue of $p$-adic completion to a map $\theta:A_\text{inf}(R)\ra R$
\[\begin{tikzcd}
A_\text{inf}(R)\arrow[r, "\theta"]\arrow[d]& R\arrow[d]\\
R^\flat\arrow[r]&R/p
\end{tikzcd}\]

From now on, we refer to integral perfectoid rings as just perfectoids. We go on now to study their algebraic properties. The proofs and motivations for these properties rely on the nice features of perfect $\F_p$-algebras. It would be helpful to record some of these properties.

\begin{remark}
Let $R$ be a perfect $\F_p$-algebra. For any ideal $I\subset R$, write $I^{[p^e]}=\{x^{p^e}\text{ }|\text{ }x\in I\}$ for any $e\in\Z_{\ge 0}$. This is the $p^e$-radical of $I$.
\begin{enumerate}
\item \todo{fill it in}
\end{enumerate}
\end{remark}

For a perfectoid $R$ we denote the \textbf{special fiber} to be the ring $\bar{R}=\dirlim_\phi R/p$. This is a perfect ring of characteristic $p$.

Perfectoid rings support a good theory of "almost mathematics", whatever that means. The key ideal to consider is the radical ideal $\sqrt{pR}$. The following lemma describes its structure.

\begin{lemma} Let $R$ be a perfectoid ring. Then
\begin{enumerate}
\item The Frobenius $\phi:R/p \ra R/p$ is surjective.
\item There exists $\pi\in R$ so that $\pi^p=p\cdot\text{unit}$ and $\ker(\phi:R/p\ra R/p)=(\pi)$.
\item The radical $\sqrt{pR}$ is flat and $(\sqrt{pR})^2=\sqrt{pR}$.
\item $R[p^\infty]=R[p]=R[\sqrt{pR}]$.
\end{enumerate}
\end{lemma}
\begin{proof}
Let $A=A_\text{inf}(R)$ and $d$ the distinguished element that is the kernel of $\theta$. (1) is clear since the Frobenius on $A/p=R^\flat$ is surjective and $R/p$ is a quotient of $A/p$. For (2), we expand $d$ into its Teichmuller expansion $d=\Sigma_{i\ge 0}[a_i]p^i$ where $a_1\in(A/p)^\times$. In particular, $d=[a_0]+pu$ where $u\in A^\times$. Hence in $R=A/(d)$, we have $[a_0]=pu$, so take $\pi=[a_0^{1/p}]$ (by multiplicativity of the Teichmuller lift).

For (3) we claim that $\sqrt{pR}=\bigcup_{n\ge 1}([a_0^{1/p^n}])$. From this it is clear that $(\sqrt{pR})^2=\sqrt{pR}$. As $\bigcup_{n\ge 1}([a_0^{1/p^n}])\subset\sqrt{pR}$ clearly, it suffices to show $R/\bigcup_{n\ge 1}([a_0^{1/p^n}])$ is reduced (i.e. the bottom is radical). But this ring is perfect, and hence reduced. \todo{check flatness}.
\end{proof}

The category of perfectoid rings is closed under pushouts in the category of all derived $p$-complete rings.

\begin{prop}[Pushouts of perfectoid rings] Let $C\leftarrow A \ra B$ be maps of perfectoids. Then the derived $p$-completion of $B\otimes^{\DerivedL}_A C$ is perfectoid (just a ring in degree 0). 
\end{prop}
\begin{proof}
Tilt the diagram and take the pushout
\[\begin{tikzcd}
A^\flat\arrow[r]\arrow[d]&B^\flat\arrow[d]\\
C^\flat\arrow[r]&R
\end{tikzcd}\]
By functoriality, the pushout $R=B^\flat\tensor_{A^\flat}C^\flat$ is perfect. Since pushouts of perfect rings have no higher homotopy groups, we have that $R\htpyeq B^\flat\tensor^{\DerivedL}_{A^\flat}C^\flat$ (derived pushouts of perfect rings are automatically discrete). Hence by derived Nakayama's lemma, $W(R)\htpyeq W(B^\flat)\widehat{\tensor}^{\DerivedL}_{W(A^\flat)}W(C^\flat)$. Now using irreducibility of distinguished elements, perform a derived $p$-complete base change along Fontaine's map $\theta:W(A^\flat)\ra A$ to conclude.
\end{proof}

The category of perfect rings enjoy nice categorical properties when embedded in the category of all rings-- it is closed under arbitrary limits and colimits. This effectively follows from functoriality of the Frobenius. However, perfectoid rings don't have such properties. We can save something by passing to derived $p$-complete rings (the above proposition allows us arbitrary colimits, and products are given). However, we can't give closure under arbitrary limits as equalizers may not exist in the category of perfectoid rings in all rings.

\begin{example} Tate's theorem gives that $\mathbb{C}_p^{\Gal(\bar{\Q_p}/\Q_p)}=\Q_p$, which is not perfectoid. As fixed points can be given as equalizers, perfectoids don't have arbitrary limits.
\end{example}

\section{structure of perfectoids}

A good class of perfectoid rings come from those which are $p$-torsionfree and $p$-torsion. We can also get "mixed" versions by taking products or fiber products. The main structure theorem for perfectoids says that that's pretty much all we can do.

\begin{prop}[Structure theorem for perfectoids] If $R$ is perfectoid, then $S=R/R[\sqrt{pR}]$ is perfectoid (and $p$-torsionfree) and
\[\begin{tikzcd}
R\arrow[r]\arrow[d]& S\arrow[d]\\
\bar{R}\arrow[r]&\bar{S}
\end{tikzcd}\]	
is a pullback diagram. So one can build any perfectoid ring using a $p$-torsionfree perfectoid ring $S$ and a perfect $\F_p$-algebra $\bar{R}$. In particular, perfectoid rings are reduced.
\end{prop}














































\end{document}