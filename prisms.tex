\documentclass[12pt]{amsproc}   
\usepackage[margin=1in, top=1in, bottom=1in]{geometry}                		
\geometry{letterpaper} 
\usepackage{amsthm}                  		
\usepackage{graphicx}	
\usepackage[mathscr]{euscript}	
\usepackage{amssymb, amsmath}
\usepackage{amscd}
\usepackage{tikz-cd}
\usepackage{xcolor}
\usepackage{csquotes}

%theorem styles
\newtheorem*{thm}{Theorem}
\newtheorem*{prop}{Proposition}
\newtheorem{lemma}{Lemma}
\newtheorem*{cor}{Corollary}

\theoremstyle{definition}
\newtheorem*{definition}{Definition}
\newtheorem*{example}{Example}
\newtheorem*{remark}{Remark}
\newtheorem*{notation}{Notation}

%notational macros
\newcommand{\A}{\mathscr{A}}
\newcommand{\sph}{\mathbb{S}}
\newcommand{\G}{\mathbb{G}}
\newcommand{\R}{\mathbb{R}}
\newcommand{\RP}{\mathbb{R}P}
\newcommand{\Q}{\mathbb{Q}}
\newcommand{\Z}{\mathbb{Z}}
\newcommand{\N}{\mathbb{N}}
\newcommand{\E}{\mathbb{E}}
\DeclareMathOperator{\pt}{pt}
\DeclareMathOperator{\BGL}{BGL}
\DeclareMathOperator{\Bcrys}{B_{crys}}
\newcommand{\eff}{\mathcal{F}}
\newcommand{\tensor}{\otimes}
\newcommand{\cotensor}{\Box}
\newcommand{\sma}{\wedge}
\newcommand{\Ztwo}{\mathbb{Z}/(2)}
\newcommand{\iso}{\cong}
\newcommand{\htpyeq}{\simeq}
\newcommand{\coproduct}{\amalg}
\newcommand{\bd}{\textbf}
\newcommand{\T}{\mathbb{T}}
\def\*{\bullet}

%arrows
\newcommand{\ra}{\rightarrow}
\newcommand{\lra}{\longrightarrow}
\newcommand{\xra}{\xrightarrow}
\newcommand{\inja}{\rightarrowtail}
\newcommand{\surja}{\twoheadrightarrow}
\newcommand{\isoarrow}{\xrightarrow{\sim}}

%new operators
\DeclareMathOperator{\Hom}{hom}
\DeclareMathOperator{\Ext}{Ext}
\DeclareMathOperator{\im}{im}
\DeclareMathOperator{\coker}{coker}
\DeclareMathOperator{\h}{H}
\DeclareMathOperator{\id}{id}
\DeclareMathOperator{\Nat}{Nat}
\DeclareMathOperator{\Spec}{Spec}
\DeclareMathOperator{\THH}{THH}
\DeclareMathOperator{\HH}{HH}
\DeclareMathOperator{\TC}{TC}
\DeclareMathOperator{\TR}{TR}
\DeclareMathOperator{\Hdr}{H_{dR}^*	}
\DeclareMathOperator{\Hcrys}{H_{crys}^*}
\DeclareMathOperator{\Het}{H_{et}^*}
\DeclareMathOperator{\Hlogcrys}{H_{log-crys}^*}
\DeclareMathOperator{\dlog}{dlog}
\DeclareMathOperator{\invlim}{\varprojlim}
\DeclareMathOperator{\holim}{holim}
\DeclareMathOperator{\dirlim}{\varinjlim}
\DeclareMathOperator{\colim}{colim}
\DeclareMathOperator{\hocolim}{hocolim}
\DeclareMathOperator{\Fil}{Fil}
\newcommand{\teich}{\underline}
\newcommand{\pcomp}[1]{{#1}^{\sma}_p}
\DeclareMathOperator{\denom}{denom}
\DeclareMathOperator{\JRad}{JRad}

%functor names
\DeclareMathOperator{\Tan}{\mathscr{T}an}
\DeclareMathOperator{\Oh}{\mathscr{O}}
\DeclareMathOperator{\F}{\mathbb{F}}
\DeclareMathOperator{\Fr}{F}
\DeclareMathOperator{\V}{V}
\DeclareMathOperator{\Der}{\mathbf{Der}}
\DeclareMathOperator{\Inf}{Inf}
\DeclareMathOperator{\gr}{gr}
\DeclareMathOperator{\Sym}{Sym}
\DeclareMathOperator{\GL}{GL}
\DeclareMathOperator{\Fun}{Fun}
\DeclareMathOperator{\Iso}{iso}
\DeclareMathOperator{\K}{K}
\DeclareMathOperator{\DerivedL}{\mathbb{L}}
\DeclareMathOperator{\Gal}{Gal}
\DeclareMathOperator{\Map}{Map}
\DeclareMathOperator{\Emb}{Emb}
\DeclareMathOperator{\DPenv}{DP-env}
\DeclareMathOperator{\logDPenv}{log-DP-env}

%category names
\newcommand{\C}{\mathscr{C}}
\newcommand{\D}{\mathscr{D}}
\newcommand{\catgrmod}{\mathscr{G}r\mathscr{M}od}
\newcommand{\catmod}{\textbf{Mod}}
\newcommand{\catcomod}{\mathscr{M}od^{\textbf{co}}}
\newcommand{\catring}{\textbf{Ring}}
\newcommand{\catalg}{\textbf{Alg}}
\newcommand{\catset}{\textbf{Set}}
\newcommand{\cataff}{\textbf{Aff}}
\newcommand{\Top}{\textbf{Top}}
\newcommand{\Mfld}{\text{Mfld}}
\newcommand{\Aff}{\textbf{Aff}}
\newcommand{\Sp}{\text{Sp}}
\newcommand{\Spc}{\text{Spc}}
\newcommand{\Ealg}[1]{\mathbb{E}_{#1}\textbf{-Alg}}
\newcommand{\Einfalg}{\mathbb{E}_{\infty}\textbf{-Alg}}
\newcommand{\catsetfin}{\textbf{Set}_\text{fin}}
\DeclareMathOperator{\Ring}{Ring}


%other macros
\newcommand{\todo}[1]{\textcolor{red}{\textbf{TODO:}\text{[#1]}}}
\newcommand{\citeme}[1]{\textcolor{blue}{\textbf{CITE ME:}\text{[#1]}}}
\newcommand{\remove}[1]{\textcolor{orange}{\textbf{REMOVE?:}\text{[#1]}}}
\newcommand{\ddd}{\cdot\cdot\cdot}


%\title{$\delta$-rings}
%\author{Calvin D. Woo}
\begin{document}
%\maketitle

\section{definition}

The category of $\delta$-pairs is formed by pairs $(A,I)$ where $A$ is a $\delta$-ring and $I\subset A$ is an ideal. Morphisms are given by relative ring maps that respect the $\delta$-structure.

Recall that we had an equivalence between the categories of perfect $\F_p$-algebras and perfect and $p$-adically complete $\delta$-rings. Significantly, this relates objects living in characteristic $p$ with characteristic 0 objects. The category of prisms will generalize this relationship to all such "perfect" objects, i.e. the perfectoid rings.

\begin{definition} A \textbf{prism} is a $\delta$-pair $(A,I)$ satisfying
\begin{enumerate}
\item $I\subset A$ defines a Cartier divisor on $\Spec A$.
\item $A$ is derived $(p,I)$-complete.
\item $p\in(I,\phi(I))$.
\end{enumerate}
\end{definition}

As before, the last statement is equivalent to saying $I$ is Zariski locally generated by distinguished elements.

\begin{definition} A map $(A,I)\ra (B,J)$ of prisms is \textbf{faithfully flat} if the map $A\ra B$ is $(p,I)$-completely flat, i.e., $A/(p,I)\ra B\otimes^{\DerivedL}_A A/(p,I)$ is flat, meaning the cotangent complex has cohomology only in degree zero.
\end{definition}

To set terminology, we'll call a prism $(A,I)$
\begin{enumerate}
\item \textit{perfect} if $A$ is perfect.
\item \textit{crystalline} if $I=(p)$.
\item \textit{bounded} if $A/I$ has bounded $p^\infty$-torsion.
\end{enumerate}

Everything we'll see will be bounded. 

\begin{example}
Some examples.

\begin{enumerate}
\item Any $p$-torsionfree and $p$-adically complete $\delta$-ring $A$ gives a bounded crystalline prism $(A,(p))$.
\item As we will see, perfect prisms = perfectoid rings.
\end{enumerate}
\end{example}

This category has a nice rigidity property:

\begin{lemma}[Rigidity] Let $(A,I)\ra (B,J)$ be a map of prisms. Then $I\tensor_A B\ra J$ is an isomorphism, so a map of prisms is determined on the underlying $\delta$-rings.
\end{lemma}
\begin{proof} Both sides are locally generated by distinguished elements. By the irreducibility of distinguished elements, we can locally conclude on $\Spec B$.
\end{proof}

\section{perfect prisms}

The most interesting class of prisms are the perfect prisms. We will show these are equivalent to the theory of perfectoid rings, and are in fact quite a bit more usable.

To begin, we give a somewhat simplified definition of a perfectoid ring.

\begin{definition} A commutative ring $R$ is \textbf{perfectoid} if it has the form $A/I$ for a perfect prism $(A,I)$. The category of perfectoid rings is the full subcategory of all commutative rings spanned by perfectoid rings.
\end{definition}

What gives? It turns out this definition coincides with that of integral perfectoids (though not with perfectoid Tate rings). 

\begin{example} Some perfectoid rings.
\begin{enumerate}
\item Let $A$ be a perfect and $p$-adically complete $\delta$-ring, $I=(p)$. Then by the structure map of such rings, $A\iso W(R)$ for a perfect $\F_p$-algebra $R$. As $(W(R),(p))$ is a crystalline prism, we have that $R\iso A/I$ is a perfectoid. So the category of perfect $\F_p$-algebras is equivalent to the category of crystalline prisms.
\item $A=\Z_p[t^{1/p^\infty}]^{\sma}_{(p,t)}$ with $\phi(t^{1/p^n})=t^{1/p^{n-1}}$ is a $p$-torsionfree $\delta$-ring, and $I=(t-p)$ is generated by a distinguished element. Hence $R=A/I=\Z_p[p^{1/p^\infty}]^{\sma}_{(p)}$ is a perfectoid ring.
\end{enumerate}
\end{example}

The main theorem of this section is

\begin{thm}[Perfect prisms$=$perfectoid rings] The mapping $(A,I)\mapsto A/I$ defines an equivalence of categories between perfect prisms and perfectoid rings.
\end{thm}

To prove this we need two helper lemmas.

\begin{lemma} Let $R$ be a perfect $\F_p$-algebra. Let $f\in R$. Then $R[f^\infty]=R[f^{1/p^n}]$ for any $n$ (hence is killed by a small power of $f$, so its an \textit{almost zero} module).
\end{lemma}
\begin{proof}
Suppose $x\in R$ such that $f^m x=0$ for some $m\ge 0$. Then $f^m x^{p^n}=0$ for all $n\ge 0$. By reducedness, $f^{m/p^n}x=0$ for any $n\ge 0$.
\end{proof}

\begin{lemma} Let $(A,I)$ be a perfect prism. Then $A/I[p^\infty]=A/I[p]$. In particular, perfect prisms are bounded.
\end{lemma}

Now we can prove the theorem.

\begin{proof}
We need to be able to recover the prism $(A,I)$ from $R=A/I$. We claim that $A\htpyeq W(R^\flat)$ where $R^\flat$ is Fontaine's tilt of $R$. As both sides are $p$-complete and perfect, it's enough to show that $A/p\htpyeq R^\flat$. 

Let $d$ be a generator of $I$. Then $R=A/(d)$ so that $R/p=A/(p,d)$. As $A/p$ is a perfect $\F_p$-algebra, we can identify the tower
\[	\cdots\ra R/p\xra{\phi} R/p\xra{\phi} R/p	\]
with the tower
\[	\cdots\ra A/(p,d^{p^2})\xra{\text{can}} A/(p,d^p)\xra{\text{can}} A/(p,d)	\]
Taking inverse limits we get
\[	(A/p)^{\sma}_{(d)}=R^\flat	\]
But $A/p$ is $d$-complete (by the first lemma).
\end{proof}











































\end{document}