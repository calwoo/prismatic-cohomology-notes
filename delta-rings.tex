\documentclass[12pt]{amsproc}   
\usepackage[margin=1in, top=1in, bottom=1in]{geometry}                		
\geometry{letterpaper} 
\usepackage{amsthm}                  		
\usepackage{graphicx}	
\usepackage[mathscr]{euscript}	
\usepackage{amssymb, amsmath}
\usepackage{amscd}
\usepackage{tikz-cd}
\usepackage{xcolor}
\usepackage{csquotes}

%theorem styles
\newtheorem*{thm}{Theorem}
\newtheorem*{prop}{Proposition}
\newtheorem{lemma}{Lemma}
\newtheorem*{cor}{Corollary}

\theoremstyle{definition}
\newtheorem*{definition}{Definition}
\newtheorem*{example}{Example}
\newtheorem*{remark}{Remark}
\newtheorem*{notation}{Notation}

%notational macros
\newcommand{\A}{\mathscr{A}}
\newcommand{\sph}{\mathbb{S}}
\newcommand{\G}{\mathbb{G}}
\newcommand{\R}{\mathbb{R}}
\newcommand{\RP}{\mathbb{R}P}
\newcommand{\Q}{\mathbb{Q}}
\newcommand{\Z}{\mathbb{Z}}
\newcommand{\N}{\mathbb{N}}
\newcommand{\E}{\mathbb{E}}
\DeclareMathOperator{\pt}{pt}
\DeclareMathOperator{\BGL}{BGL}
\DeclareMathOperator{\Bcrys}{B_{crys}}
\newcommand{\eff}{\mathcal{F}}
\newcommand{\tensor}{\otimes}
\newcommand{\cotensor}{\Box}
\newcommand{\sma}{\wedge}
\newcommand{\Ztwo}{\mathbb{Z}/(2)}
\newcommand{\iso}{\cong}
\newcommand{\htpyeq}{\simeq}
\newcommand{\coproduct}{\amalg}
\newcommand{\bd}{\textbf}
\newcommand{\T}{\mathbb{T}}
\def\*{\bullet}

%arrows
\newcommand{\ra}{\rightarrow}
\newcommand{\lra}{\longrightarrow}
\newcommand{\xra}{\xrightarrow}
\newcommand{\inja}{\rightarrowtail}
\newcommand{\surja}{\twoheadrightarrow}
\newcommand{\isoarrow}{\xrightarrow{\sim}}

%new operators
\DeclareMathOperator{\Hom}{hom}
\DeclareMathOperator{\Ext}{Ext}
\DeclareMathOperator{\im}{im}
\DeclareMathOperator{\coker}{coker}
\DeclareMathOperator{\h}{H}
\DeclareMathOperator{\id}{id}
\DeclareMathOperator{\Nat}{Nat}
\DeclareMathOperator{\Spec}{Spec}
\DeclareMathOperator{\THH}{THH}
\DeclareMathOperator{\HH}{HH}
\DeclareMathOperator{\TC}{TC}
\DeclareMathOperator{\TR}{TR}
\DeclareMathOperator{\Hdr}{H_{dR}^*	}
\DeclareMathOperator{\Hcrys}{H_{crys}^*}
\DeclareMathOperator{\Het}{H_{et}^*}
\DeclareMathOperator{\Hlogcrys}{H_{log-crys}^*}
\DeclareMathOperator{\dlog}{dlog}
\DeclareMathOperator{\invlim}{\varprojlim}
\DeclareMathOperator{\holim}{holim}
\DeclareMathOperator{\dirlim}{\varinjlim}
\DeclareMathOperator{\colim}{colim}
\DeclareMathOperator{\hocolim}{hocolim}
\DeclareMathOperator{\Fil}{Fil}
\newcommand{\teich}{\underline}
\newcommand{\pcomp}[1]{{#1}^{\sma}_p}
\DeclareMathOperator{\denom}{denom}
\DeclareMathOperator{\JRad}{JRad}

%functor names
\DeclareMathOperator{\Tan}{\mathscr{T}an}
\DeclareMathOperator{\Oh}{\mathscr{O}}
\DeclareMathOperator{\F}{\mathbb{F}}
\DeclareMathOperator{\Fr}{F}
\DeclareMathOperator{\V}{V}
\DeclareMathOperator{\Der}{\mathbf{Der}}
\DeclareMathOperator{\Inf}{Inf}
\DeclareMathOperator{\gr}{gr}
\DeclareMathOperator{\Sym}{Sym}
\DeclareMathOperator{\GL}{GL}
\DeclareMathOperator{\Fun}{Fun}
\DeclareMathOperator{\Iso}{iso}
\DeclareMathOperator{\K}{K}
\DeclareMathOperator{\Gal}{Gal}
\DeclareMathOperator{\Map}{Map}
\DeclareMathOperator{\Emb}{Emb}
\DeclareMathOperator{\DPenv}{DP-env}
\DeclareMathOperator{\logDPenv}{log-DP-env}

%category names
\newcommand{\C}{\mathscr{C}}
\newcommand{\D}{\mathscr{D}}
\newcommand{\catgrmod}{\mathscr{G}r\mathscr{M}od}
\newcommand{\catmod}{\textbf{Mod}}
\newcommand{\catcomod}{\mathscr{M}od^{\textbf{co}}}
\newcommand{\catring}{\textbf{Ring}}
\newcommand{\catalg}{\textbf{Alg}}
\newcommand{\catset}{\textbf{Set}}
\newcommand{\cataff}{\textbf{Aff}}
\newcommand{\Top}{\textbf{Top}}
\newcommand{\Mfld}{\text{Mfld}}
\newcommand{\Aff}{\textbf{Aff}}
\newcommand{\Sp}{\text{Sp}}
\newcommand{\Spc}{\text{Spc}}
\newcommand{\Ealg}[1]{\mathbb{E}_{#1}\textbf{-Alg}}
\newcommand{\Einfalg}{\mathbb{E}_{\infty}\textbf{-Alg}}
\newcommand{\catsetfin}{\textbf{Set}_\text{fin}}
\DeclareMathOperator{\Ring}{Ring}


%other macros
\newcommand{\todo}[1]{\textcolor{red}{\textbf{TODO:}\text{[#1]}}}
\newcommand{\citeme}[1]{\textcolor{blue}{\textbf{CITE ME:}\text{[#1]}}}
\newcommand{\remove}[1]{\textcolor{orange}{\textbf{REMOVE?:}\text{[#1]}}}
\newcommand{\ddd}{\cdot\cdot\cdot}


%\title{$\delta$-rings}
%\author{Calvin D. Woo}
\begin{document}
%\maketitle

This is a set of notes on the theory of $\delta$-rings. Fix a prime $p$ for the rest of the note. The running slogan is that $\delta$-rings are "rings with a lift of Frobenius modulo $p$". In fact, if we take this slogan literally, it becomes a definition. Note that if $A$ is a commutative ring equipped with a map $\phi:A\ra A$ that is a lift of Frobenius on $A/p$, then for each $x\in A$ we have
\[	\phi(x) = x^p + p\delta(x)	\]
for some map of sets $\delta:A\ra A$. If $A$ is $p$-torsion free, then $\delta$ is \textit{uniquely} determined by this formula. The definition relations for $\delta(-)$ then come from the relations encoding the fact that $\phi(-)$ is a ring homomorphism.

\begin{definition} A $\delta$-\textbf{ring} is a pair $(A,\delta)$ where $A$ is a commutative ring, $\delta:A\ra A$ a map of sets with $\delta(0)=\delta(1)=0$ and satisfying the identities
\[	\delta(x+y)=\delta(x)+\delta(y)+\frac{x^p+y^p-(x+y)^p}{p}\]
and
\[	\delta(xy)=x^p\delta(y)+y^p\delta(x)+p\delta(x)\delta(y)	\]
\end{definition}

In the literature, a $\delta$-structure is sometimes called a $p$-derivation (especially by those who like $p$-adic differential equations). In computations it is often useful to write the multiplicative relation for $\delta$ in the asymmetric form
\[	\delta(xy) = \phi(x)\delta(y)+y^p\delta(x)\]

A cute property of the Frobenius lift is that it is a homomorphism of $\delta$-rings.

\begin{lemma} Let $A$ be a $\delta$-ring. Then $\phi:A\ra A$ is a $\delta$-map, that is, $\phi(\delta(x))=\delta(\phi(x))$ for all $x\in A$.
\end{lemma}
\begin{proof} Suppose $A$ is $p$-torsion free. Then we have $\delta(x)=\frac{1}{p}(\phi(x)-x^p)$. We can then verify directly that this is the case. Expanding,
\[	\phi(\delta(x))=\frac{1}{p}(\phi^2(x)-\phi(x)^p)=\delta(\phi(x))	\]
When $A$ has $p$-torsion, we can lift this argument to a $p$-torsionfree $\delta$-ring using free functors.
\end{proof}

A $p$-torsionfree ring with a lift of Frobenius $\phi$ furnishes an example of a $\delta$-ring. Since we have plenty of examples of this, we get plenty of examples of $\delta$-rings.

\begin{example} 
Some examples of $\delta$-rings.

\begin{enumerate}
\item $\Z$ with $\phi=\id$. Here, $\delta$ is pretty explicit, given by $\delta(n)=\frac{1}{p}(n-n^p)$. This is the initial object in the category of $\delta$-rings.
\item The polynomial ring $\Z[x]$ has Frobenius lift $\phi$ determined by $\phi(x)=x^p + pg(x)$ for any $g(x)\in \Z[x]$.
\item For any perfect field $k$ of characteristic $p>0$, the ring of Witt vectors $W(k)$ with $\phi$ given by the standard (unique!) Frobenius lift. Since there is only one such lift, $W(k)$ only admits a $\delta$-structure in one way. Geometrically, this corresponds to the $p$-cotangent space of $W(k)$ being trivial.
\end{enumerate}
\end{example}

In an obvious way, $\delta$-rings form a category, which we'll call $\Ring^\delta$. Here the $p$ is implicit, and hence suppressed. A thing we would like is to understand whether $\Ring^\delta$ has any nice categorical properties. It is cumbersome to prove them with all these relations floating about, so we want a more principled way to talk about $\delta$-rings.

\begin{prop} A $\delta$-structure on a ring $A$ is the same as a ring section $w:A\ra W_2(A)$ of the map $e:W_2(A)\ra A$ that forgets the second component.
\end{prop}
\begin{proof}
We can write the truncated Witt vector $W_2(A)$ as the fiber product of
\[\begin{tikzcd}
&&A\arrow[d, "can"]\\
A\arrow[r, "can"]&A/p\arrow[r, "\phi"]& A/p
\end{tikzcd}\]
Then it becomes clear that a ring map $A\ra W_2(A)$ that splits the projection of $W_2(A)$ to $A$ is the same data as a Frobenius lift on $A$.
\end{proof}

The functoriality of $W_2(-)$ shows that $\Ring^\delta$ is a bicomplete category, i.e., has all limit and colimits. Combining this with the adjoint functor theorem gives the useful fact that the forgetful functor $\Ring^\delta\ra \Ring$ has both a left and right functor-- the left adjoint is the \textbf{free $\delta$-ring construction}, and the right adjoint is given by the Witt vector construction.

We give a description of the free $\delta$-ring on a single variable: $\Z\{x\}$ is the polynomial ring $\Z[x_0,x_1,x_2,...]$ with $x=x_0$ and $\delta(x_i)=x_{i+1}$. In particular, the free $\delta$-rings are $p$-torsion free. This, combined with the existence of all colimits give us a way to build $\delta$-rings using generators and relations. It's a good category indeed.

\subsection{some ring-theoretic properties}
$\delta$-rings are stable under some natural ring-theoretic operations like localization and quotients.

\begin{lemma}[Localizations of $\delta$-rings] Let $A$ be a $\delta$-ring and $S\subset A$ a multiplicative subset stable under the Frobenius lift. Then there is a unique $\delta$-structure on the localization $S^{-1}A$ extending the one on $A$.
\end{lemma}
\begin{proof}\todo{prove this}\end{proof}

\begin{lemma}[Completions of $\delta$-rings] Let $A$ be a $\delta$-ring and $I\subset A$ a finitely generated ideal that contains $p$. Then there is a unique $\delta$-structure on the $I$-adic completion $A^{\sma}_I$.
\end{lemma}
\begin{proof}\todo{prove this}\end{proof}

As a consequence, $\Z_p$ with the $\delta$-structure given by the identity $\phi=\id$ is the initial object in the category of $p$-adically complete $\delta$-rings.

\begin{lemma}[Etale extensions of $\delta$-rings] Fix a map $A\ra B$ of $p$-adically complete and $p$-torsion free rings, where $A$ is a $\delta$-ring. Suppose $A\ra B$ is etale modulo $p$. Then $B$ has a unique $\delta$-structure compatible with the one on $A$.
\end{lemma}
\begin{proof} By $p$-torsionfreeness of the two rings, we just need to exhibit a Frobenius lift on $B$ compatible with the one on $A$. By $p$-adic completeness, we just need to exhibit this Frobenius lift modulo $p^n$ for all $n\ge 1$.

But then we do this by induction. When $n=1$, this is given by the pushout of the Frobenius on $A$, giving the relative Frobenius on $B$. For $n>1$, we appeal to the topological invariance of the etale site.
\end{proof}

\begin{lemma}[Quotients of $\delta$-rings] Let $A$ be a $\delta$-ring and $I\subset A$ an ideal such that $\delta(I)\subset I$ (stable under $\delta$). Then there is a unique $\delta$-structure on the quotient $A/I$ compatible with the one on $A$.
\end{lemma}
\begin{proof} Suffices to show that for $x\in A$ and $c\in I$, $\delta(x+c)\equiv\delta(x)\mod{I}$. But this follows from the additivity relation that $\delta$-rings must satisfy.
\end{proof}

\subsection{perfect $\delta$-rings}
To get us closer to the theory of prismatic cohomology, we will study an important class of $\delta$-rings-- the perfect ones.

Just like for regular rings, we'll say a $\delta$-ring $A$ is \textbf{perfect} if the Frobenius $\phi:A\ra A$ is an isomorphism. Perfect rings have incredible algebraic properties, but arguably the genesis of these traits come from the following derived fact:

\begin{thm} Let $A$ be a perfect $\F_p$-algebra. Then the cotangent complex $L_{A/\F_p}$ vanishes. \end{thm}
\begin{proof} Note that in characteristic $p>0$, the derivative of the Frobenius on a polynomial ring is given by
\[	d\phi(x)=d(x^p)=px^{p-1}dx=0	\]
The Frobenius is functorial, so taking a functorial simplicial resolution of $A$ by polynomial $\F_p$-algebras gives that the Frobenius induces the 0 map on the cotangent complex. But since it is also simultaneously an isomorphism by perfection, the cotangent complex must vanish $L_{A/\F_p}\htpyeq 0$.
\end{proof}

\begin{remark} As the deformation theory of $A$ is controlled by its cotangent complex, the above result says that there are no obstructions to lifting over perfect $\F_p$-algebras. As a consequence, we get the equivalence of the following categories:
\begin{enumerate}
\item Perfect $\F_p$-algebras.
\item The category of flat $\Z/p^n$-algebras $\bar{A}$ with $\bar{A}/p$ perfect.
\item $p$-adically complete and $p$-torsionfree $\Z_p$-algebras $\bar{A}$ with $\bar{A}/p$ perfect.
\end{enumerate}
The equivalence of (1) and (2) follow from deformation theory, and that of (2) and (3) follow from the characterization of the category of $p$-adically complete and $p$-torsionfree $\Z_p$-algebras as the inverse limit of the categories of flat $\Z/p^n$-algebras.
\end{remark}

To relate this to perfect $\delta$-rings, we use the following algebraic fact.

\begin{lemma} Let $A$ be a $\delta$-ring and let $x\in A$ with $px=0$. Then $\phi(x)=0$. In particular, if $\phi$ is injective, then $A$ is $p$-torsionfree. \end{lemma}
\begin{proof}
In $A[1/p]$, we would have trivially $x=0$ so $\phi(x)=0$, so we assume that $A$ is a $\Z_{(p)}$-algebra. Expanding $\delta(px)$ we have
\[	0 = \delta(px)=p^p\delta(x)+\phi(x)\delta(p)	\]
Note that in $\Z_{(p)}$, we see that
\[	\delta(p)=\frac{\phi(p)-p^p}{p}=\frac{p-p^p}{p}=1-p^{p-1}\in\Z^*_{(p)}	\]
is a unit, so it suffices to show that $p^p\delta(x)=0$. But
\[	p^p\delta(x)=p^{p-1}\cdot p\delta(x)=p^{p-1}(\phi(x)-x^p)=p^{p-2}(\phi(px)-px\cdot x^{p-1})=0	\]
where we used that $px=0$ in the end.
\end{proof}

This leads to a classification of perfect $\delta$-rings: they just come from perfect $\F_p$-algebras!

\begin{thm} The category of perfect $\delta$-rings that are $p$-adically complete is equivalent to the category of perfect $\F_p$-algebras.
\end{thm}
\begin{proof}
The above lemma shows that perfect $\delta$-rings are always $p$-torsionfree, and so we have a forgetful functor from the category of perfect $p$-adically complete $\delta$-rings to the category of $p$-adically complete and $p$-torsionfree $\Z_p$-algebras $\bar{A}$ with $\bar{A}/p$ perfect. By the above remark, this is equivalent to the category of perfect $\F_p$-algebras. To go back to perfect $\delta$-rings, fix a perfect $\F_p$ algebra, and by deformation theory lift it to a $p$-adically complete and $p$-torsionfree ring. As the cotangent complex vanishes, such a lift is unique and is equipped with a unique lift of Frobenius. By $p$-torsionfreeness of the lift, this defines the required $\delta$-structure.
\end{proof}


\begin{remark}
We can be explicit about the equivalence above. One functor is given by modulo $p$, $A\mapsto A/p$, while the other is given by the Witt vector construction $A\mapsto W(A)$. In other words, every $p$-adically complete perfect $\delta$-ring has the form $W(R)$ for some perfect $\F_p$-algebra $R$.
\end{remark}

\subsection{distinguished elements}

From now on until specified otherwise, we assume that any commutative ring we see now are $p$-local, i.e. $p\in\JRad(A)$, where $\JRad(A)$ is the Jacobson radical of $A$. In particular, any $p$-adically complete ring is so.

In the previous sections, the fact that $\delta(p)$ was a unit came in handy in relating perfect $\delta$-rings to perfect $\F_p$-algebras. It would do us well to give a name to elements with this property.

\begin{definition} Let $R$ be a $\delta$-ring. An element $d\in R$ is called \textbf{distinguished} if $\delta(d)$ is a unit. (This terminology dates back to Fontaine).
\end{definition}

The slogan is that a distinguished element is a "deformation" of $p$. As ring homomorphisms preserve units, we note that any morphism of $\delta$-rings preserves distinguished elements. In fact, it just needs to commute with $\delta$: the Frobenius lift $\phi$ commutes with $\delta$, and so we see that $\phi$ preserves distinguished elements. Even better, as our rings are $p$-local, if $\phi(d)$ is distinguished so is $d$.

\begin{example}
Here are some examples of distinguished elements in $\delta$-rings.
\begin{enumerate} 
\item (\textit{Crystalline cohomology}) Let $A=\Z_{(p)}$ with $d=p$. More generally, for any $\delta$-ring $A$ with $p\in\JRad(A)$, the image of $p\in A$ is distinguished.
\item (\textit{Breuil-Kisin cohomology}) Fix a discretely-valued extension of $K/\Q_p$ with uniformizer $\pi$. Let $W(k)\subset\Oh_K$ be the maximal unramified subring. Let $A=W(k)[\![u]\!]$ with $\delta$-structure induced by the canonical one on $W(k)$ and satisfying $\phi(u)=u^p$. There is a $W(k)$-equivariant surjection $A\ra\Oh_K$ determined by $u\mapsto\pi$. Any generator of the kernel of this map is distinguished.
\end{enumerate}
\end{example}

\begin{example} We construct a universal $p$-local $\delta$-ring $A$ with a distinguished element $d\in A$. Take the free $\delta$-ring on an element $\Z\{d\}$. Maps out of this classify $\delta$-rings with a choice of \textit{any} element in it. To ensure that the image of $\delta(d)$ is  unit, we can freely adjoin its inverse to our free $\delta$-ring to get $\Z_{(p)}\{d,\delta(d)^{-1}\}$. But this can be written as the localization $S^{-1}\Z_{(p)}\{d\}$ along the multiplicative set $S=\{\delta(d),\phi(\delta(d)),\phi^2(\delta(d)),...\}$. This ring may no longer be $p$-local, so for extra measure we $p$-localize again.
\end{example}

An important class of distinguished elements can be characterized when we are in the context of a perfect $p$-adically complete $\delta$-ring. In this case we have seen that such rings are $W(R)$ for some perfect $\F_p$-algebra $R$, and so its elements have Teichmuller expansions.

\begin{lemma} Let $R$ be a perfect $\F_p$-algebra. Then an element $d\in W(R)$ is distinguished if and only if the coefficient of $p$ in its Teichmuller expansion is a unit. All such distinguished elements are thus nonzero-divisors.
\end{lemma}
\begin{proof}
Fix an element $d$ and its Teichmuller expansion $\sum_{i\ge 0} [a_i]p^i$. From the formula $\delta(d)=\frac{\phi(d)-d^p}{p}$ and the formula for the Frobenius $\phi([a_0])=[a_0^p]$, we get that mod $p$, $\delta(d)\equiv [a_1^p]$. As $W(R)$ is $p$-adically complete, it follows that $\delta(d)$ is a unit precisely when $a_1\in R$ is a unit.

To see the second claim, we consider the Teichmuller expansion of $xd=0$, where $x$ is an element killing $d$.
\end{proof}














\end{document}